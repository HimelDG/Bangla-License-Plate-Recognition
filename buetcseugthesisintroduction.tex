\chapter{Introduction}\label{intro}


\section{Automatic Bangla License Plate Recognition}
Automatic Bangla License Plate Recognition(ALPR)  means detecting bangla languager characters from a license plate of any vehicle without human intervention. In this character recognition process we have follow the procedures given below.
\begin{itemize}
\item Reduction of noise from already captured car images.
\item Feature extraction.
\item Creation of data set from the features.
\item Recognition of characters.
\end{itemize}

Distorted detection of License Plate(LP) makes this procedure difficult. To extract the correct feature value from the image, some preprocesses need to be handled. Some significant issues such as,
\begin{itemize}

\item Straighten the image
\item Smoothing the image
\item Reduction of noise
\end{itemize}
\section{Applications of ALPR}
In an ALPR system, stationary cameras are mounted on road signs, street lights, buildings or highway  overpass for capturing images of moving or parked vehicles. Then the captured images will go through a software system that will first detect the LP location in the image  and then using the image, will extract some features values, and then using the features values in some machine learning algorithms finally get the final readings of that vehicle license plate. The recovered identity of the vehicle can be used in real time or stored in the database for future use.

One of the most important applications of ALPR system is traffic law enforcement. ALPR system can be used for automatic and faster identification of stolen vehicles, criminal cars, speeder and traffic law breakers. Another great approach is automatic toll collection for highways, flyovers and bridges. This automatic approach saves time and reduce traffic congestion during toll collection. Even is gas station, shopping  mall parking ALPR can be very handy to managing car access. Many developed countries are using ALPR for this kind of purposes to make their system more automated and make day to day life of people more easier. 


\section{Background and Motivation}
ALPR is becoming popular from the past decades because of it's various real-life applications. Mainly a ALPR system consists of three important steps. Image acquisition, LP detection, feature extraction and character recognition. Among this three steps feature extraction and character recognition is totally software dependent and have to face many technical issues. The image acquired from the previous is not always suitable for feature extraction and character recognition. So, some preprocessing has to be done. Otherwise accuracy of character recognition will not be that satisfactory.
So, far many algorithm and features has been implemented for optical character recognition. But all the algorithm and features have some flaws and weakness. Most of the algorithms have not considered the distorted, noisy ,titled, low contrasted images that much. That's why particular afford is needed to consider those problems and hence make such feature to make the most accuracy of character recognition.

				
\section{Objective}
In this thesis our objective was to extract feature values from already detected LP and then using machine language algorithm to detect the characters. To develop an effective ALPR, we considered the issue of hazardous image background, low contrast image and horizontal tilt problem. Specific objective if this thesis is stated below.
\begin{itemize}
\item Automatic feature extraction and recognition of Bangla characters from noiseless clean printed LP images.
\item Automatic feature extraction and recognition of Bangla character from distorted, low contrast images.
\item Automated character recognition from Bangla handwritten dataset.
\end{itemize}

\section{Scope of the Thesis}
\begin{itemize}
\item In this thesis we focus only on feature extraction and character recognition. We are not concern about detecting license plate from the vehicle image.
\item In case of noisy image we try to remove noise as far as possible as the detection if image was not in our concern. Too much noisy and distorted images are not considered.
\item We try to implement those features only which we think suitable for our images of vehicles.
\item For image segmentation we initially segmented the character portion and the digits portion.
\item Images where character portion and digit portion got overlapped wasn't considered. 
\end{itemize}
\section{Outline of Thesis}
The rest of the thesis is organized as follows:
In the Chapter 2, we briefly describe several feature extraction techniques and algorithms that we used for our ALPR system. Also the literature review of several existing ALPR techniques are given in this chapter.
\endinput